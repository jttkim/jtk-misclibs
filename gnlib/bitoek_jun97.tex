\documentclass[a4paper, fleqn]{article}

\usepackage[T1]{fontenc}
\usepackage[german]{babel}

\sloppy


\begin{document}

\section{Evolutionsmodelle und \texttt{gnlib}}

\texttt{gnlib} ist eine \textsl{library}, die Funktionen zum Erfassen,
Speichern, Laden und Visualisieren von Stammb"aumen in Computersimulationen
von Evolutionsprozessen zur Verf"ugung stellt. Um Computersimulationen
mit \texttt{gnlib} zu untersuchen, sind einige, wenige Modifikationen
des Simulationsprogramms erforderlich. Das Programm kann dann zu jedem
gew"unschten Zeitpunkt den Stammbaum der Individuen in unterschiedlichen
Formaten abspeichern. An Formaten stehen hier das New Hampshire format,
das bei Software zur Rekonstruktion von Stammb"aumen ein g"angiger Standard
ist, sowie ein \texttt{gnlib}-eigenes Format zur Verf"ugung. Zur
Weiterverarbeitung der im \texttt{gnlib}-Format gespeicherten Stammb"aume
stehen verschiedene tools zur Verf"ugung. Weiterhin werden durch \texttt{gnlib}
einige Funktionen zur online-Analyse des phylogenetischen Prozesses
bereitgestellt.


\subsection{Erweiterung von Computersimulationen mit \texttt{gnlib}}
\label{function_overview}

Folgende Ver"anderungen am Programm sind f"ur Untersuchungen mit \texttt{gnlib}
erforderlich:

\begin{enumerate}

\item F"ur jeden Evolutionsproze"s mu"s ein \textsl{struct} des Typs
    \verb|GN_TREE| angelegt werden. Vor der ersten Verwendung dieser
    Struktur mu"s sie mittels der Funktion \verb|gn_init_tree|
    initialisiert werden.

\item F"ur jedes Individuum mu"s ein \textsl{struct} des Typs
    \verb|GN_NODE_ID| angelegt werden. In dieser Struktur wird
    durch die \texttt{gnlib} eine Identifikation (im Folgenden auch als
    ID bezeichnet) f"ur das entsprechende
    Individuum angelegt, anhand derer das Individuum bei sp"ateren
    Aufrufen von \texttt{gnlib}-Funktionen wiedererkannt wird.

\item Bei jeder Erzeugung eines neuen Individuums mu"s die Funktion
    \verb|gn_new_treenode| aufgerufen werden. Durch diesen Aufruf wird
    das neu erzeugte Individuum in den Stammbaum "`eingebaut"'. Der
    Stammbaum, zu dem das Individuum geh"ort, sein Elter sowie sein
    Genom werden bei diesem Aufruf als Argumente "ubergeben.

\item Beim jedem Absterben eines Individuums mu"s die Funktion
    \verb|gn_node_death| aufgerufen werden.

\item Beim Abspeichern des Simulationszustands m"ussen sowohl der
    Stammbaum sowie die IDs aller abgespeicherten Individuen mitgesichert
    werden. Diese Information mu"s dann bei der Fortsetzung des
    Simulations aus dem entsprechende File wieder eingelesen werden.
    Zur Realisierung dieses Vorgehens in C++ werden die Operatoren
    \verb|<<| und \verb|>>| f"ur die genannten Datentypen
    \verb|GN_TREE| und \verb|GN_NODE_ID| bereitgestellt.

\end{enumerate}

Diese Funktionen werden im Folgenden n"aher erl"autert.


\subsubsection{\texttt{gn\_init\_tree}}
\label{gn_init_tree}

\begin{verbatim}
extern void gn_init_tree(GN_TREE *gtree);
\end{verbatim}

Diese Funktion initialisiert die \verb|GN_TREE| Struktur so, da"s sie
einen leeren Stammbaum enth"alt. Das Argument \verb|gtree| ist ein
\textsl{pointer} auf die zu initialisierende Struktur.
\verb|gn_init_tree| sollte f"ur jede \verb|GN_TREE| Struktur aufgerufen
werden, bevor sie zum ersten Mal mit den in den folgenden Abschnitten
beschriebenen Funktionen verwendet wird.

Eine einmal verwendete \verb|GN_TREE| Struktur darf nicht mit
\verb|gn_init_tree| initialisiert werden, bevor der durch den an dieser Struktur
"`h"angenden"' Stammbaum belegte Speicher mit \verb|gn_free_tree|
(s.\ \ref{gn_free_tree}) freigegeben wurde.

Unmittelbar nach dem Aufruf von \verb|gn_init_tree||, in jedem Fall aber
vor der ersten Verwendung der initialisierten Struktur, kann die Strategie zum
Entfernen von "abgestorbenen Teilen" des Stammbaums, also von Teilb"aumen,
die keine rezenten Individuen mehr enthalten, festgelegt werden. Standardm"a"sig
werden alle Knoten aus dem Baum entfernt, die zu nicht mehr lebenden Individuen
geh"oren und die nur noch mit einem Teilbaum, der zu lebenden Individuen
geh"orende Knoten enth"alt, verbunden sind. Diese Knoten werden als
"`Monolinks"' bezeichnet. Durch das Entfernen solcher Monolinks werden nur
Knoten, bei denen eine Bifurkation der Genealogie vorliegt, erhalten; eine
Rekonstruktion der Gesamtmenge aller Vorfahren eines rezenten Individuums
ist nicht m"oglich. Dieses Vorgehen stellt jedoch sicher, da"s die Anzahl
der im Speicher befindlichen Knoten das Doppelte der aktuellen Populationsgr"o"se
nicht "uberschreiten kann.

Soll die M"oglichkeit zur Extraktion vollst"andiger genealogischer Linien
erhalten werden, mu"s vor der Verwendung der initialisierten \verb|GN_TREE|
Struktur die Komponente \verb|pruning_strategy| auf den konstanten Wert
\verb|GN_PRUNE_DEAD| gesetzt werden. Bei dieser Strategie ist mit einem
Anwachsen der Anzahl im Hauptspeicher gehaltenen Knoten im Verlauf der
Simulation zu rechnen.


\subsubsection{\texttt{gn\_free\_tree}}
\label{gn_free_tree}

\begin{verbatim}
extern int gn_free_tree(GN_TREE *gtree);
\end{verbatim}

Durch diese Funktion werden alle Speicherbl"ocke, die durch Knoten der
"ubergebenen \verb|GN_TREE| Struktur belegt werden, wieder
freigegeben.  Das Argument \verb|gtree| ist ein \textsl{pointer} auf
Struktur, deren Knoten gel"oscht werden sollen.  Das \textsl{struct}
selbst wird nat"urlich nicht mit \verb|free()| freigegeben, so da"s
\textsl{structs}, die als lokale oder globale Variablen definiert
wurden, verwendet werden k"onnen.

Nach dem Aufruf von \verb|gn_free_tree| d"urfen keine \verb|GN_NODE_ID|
Strukturen, die zu dem gel"oschten Baum geh"oren, mehr bei weiteren
\texttt{gnlib}-Aufrufen verwendet werden.


\subsubsection{\texttt{gn\_new\_treenode}}
\label{gn_new_treenode}

\begin{verbatim}
extern int gn_new_treenode(GN_TREE *gtree,
        const GN_NODE_ID *p_ancestor_id, long generation,
        const char *name, const char *genome,
        GN_NODE_ID *new_nid);
\end{verbatim}

Mit dieser Funktion wird ein neuer Knoten in einen bestehenden Stammbaum
eingesetzt. Die Argumente sind:

\begin{description}

\item[gn\_tree:] \textsl{Pointer} auf den Stammbaum (genaugenommen: das
    \verb|GN_TREE| \textsl{struct}, das den Stammbaum beschreibt), zu dem
    ein Individuum hinzugef"ugt werden soll.

\item[p\_ancestor\_id:] \textsl{Pointer} auf die ID des Elternindividuums.
    Dieses Argument darf den Wert \verb|NULL| haben, in diesem Fall wird
    das Individuum als Urahn eines neuen Teilbaums betrachtet. Dies ist
    typischerweise f"ur Individuen, die beim Start eines Simulationslaufs
    erzeugt werden, sinnvoll.

\item[generation:] Dieses Argument gibt den Zeitschritt (innerhalb der Simulation)
    an, in dem das Individuum erzeugt wird. Da \verb|generation| in der
    vorliegenden Implementation den Typ \verb|long| hat, ist die \texttt{gnlib}
    derzeit nur f"ur Simulationen mit ganzzahligen, diskreten Zeitschritten
    geeignet.

\item[name:] Hier kann ein Name f"ur das Individuum angegeben werden, der bei
    der Ausgabe und Visualisierung der Stammb"aume verwendet werde soll.
    Wird hier ein Nullpointer oder ein Leerstring "ubergeben, generieren die
    \texttt{gnlib}-Funktionen bei Bedarf einen Namen aus der \verb|GN_NODE_ID|.
    Die L"ange des Namens darf maximal die Konstanten \verb|GN_MAX_NODENAME_LENGTH|
    erreichen.

\item[genome:] Hier kann ein beliebig langer, nullterminierter String "ubergeben
    werden, der das Genom, oder die relevante genetische Konstitution oder
    Vergleichbares beschreibt.

\item[new\_nid:] \textsl{Pointer} auf eine \verb|GN_NODE_ID| Struktur, in der
    die ID des neuen Individuums angelegt wird. Diese Struktur mu"s vom
    aufrufenden Programm bereitgestellt werden, und ein \textsl{Pointer}
    auf diese Struktur mu"s bei Aufrufen von \verb|gn_new_treenode| (wenn das
    neu erzeugte Individuum sich seinerseits vermehrt) und beim Aufruf von
    \verb|gn_node_death| (s.\ \ref{gn_node_death}) beim Absterben des Individuums
    "ubergeben werden.

\end{description}

\verb|gn_new_treenode| sollte f"ur jedes Individuum aufgerufen werden, das
tats"achlich in die Population aufgenommen wird. Ein Aufruf beim Erzeugen von
Strukturen, die zur tempor"aren Zwischenspeicherung von Individuen oder
Vergleichbarem dienen, sollte dagegen vermieden werden. F"ur jedes Individuum,
f"ur das mit \verb|gn_new_treenode| ein Stammbaumknoten angelegt wurde, sollte
sp"ater auch \verb|gn_node_death| (s.\ \ref{gn_node_death}) aufgerufen werden,
um akkurate Stammb"aume zu erhalten und um eine unn"otige Belastung der
Speicherresourcen zu vermeiden.


\subsubsection{\texttt{gn\_node\_death}}
\label{gn_node_death}

\begin{verbatim}
extern int gn_node_death(GN_TREE *gtree, const GN_NODE_ID *gn_nid,
        long death_generation);
\end{verbatim}

Diese Funktion mu"s aufgerufen werden, wenn ein Individuum, zu dem mit
\verb|gn_new_treenode| ein Knoten im Stammbaum erzeugt wurde, stirbt. Die
Argumente sind:

\begin{description}

\item[gn\_tree:] \textsl{Pointer} auf den Stammbaum, zu dem das gestorbene
    Individuum geh"ort.

\item[gn\_nid:] \textsl{Pointer} auf die ID des gestorbenen Individuums.

\item[generation:] Dieses Argument gibt den Zeitschritt (innerhalb der Simulation)
    an, in dem das Individuum stirbt. Anhand dieses Arguments berechnet
    die \texttt{gnlib} die Spanne, die das Individuum "uberlebt hat.

\end{description}

Der Aufruf von \verb|gn_node_death| verursacht (abh"angig von der
\verb|pruning_strategy| des Stammbaums) die Freigabe des Speichers
des Knotens. Daher darf die Knoten-ID eines Individuums nach dem Aufruf
von \verb|gn_node_death| nicht mehr als Eltern-ID bei \verb|gn_new_treenode|
"ubergeben werden.


\subsubsection{\texttt{gn\_tree\_height}}
\label{gn_tree_height}

\begin{verbatim}
extern long gn_tree_height(const GN_TREE *gtree, long generation);
\end{verbatim}

Diese Funktion berechnet die H"ohe des Stammbaums, also den zeitlichen Abstand
zwischen der am weitesten zur"uckliegenden genealogischen Bifurkation und
der aktuellen Zeit. Die Argumente sind:

\begin{description}

\item[gn\_tree:] \textsl{Pointer} auf den Stammbaum, dessen H"ohe ermittelt werden
    soll.

\item[generation:] Aktueller Zeitschritt.

\end{description}


\subsubsection{\texttt{gn\_tree\_length}}
\label{gn_tree_length}

\begin{verbatim}
extern long gn_tree_length(const GN_TREE *gtree, long generation);
\end{verbatim}

Diese Funktion berechnet die L"ange des Stammbaums; diese ist definiert als
die Summe aller Kanten im Stammbaum, wobei die zur untersten Bifurkation
hinf"uhrenden Kanten nicht mitgez"ahlt werden. Die Argumente sind wie bei
\verb|gn_tree_height| (s.\ \ref{gn_tree_height}).


\subsubsection{\texttt{gn\_ddc}}
\label{gn_ddc}

\begin{verbatim}
extern double gn_ddc(const GN_TREE *gtree, long generation);
\end{verbatim}

Diese Funktion berechnet die Distanzverteilungskomplexit"at der rezenten
Population, wobei als Distanzma"s die phylogenetische Distanz, also die
Zeit, die seit der Bifurkation der zu zwei unterschiedlichen, rezenten Arten
hinf"uhrenden Teilb"aumen vergangen ist. Die Argumente sind wie bei
\verb|gn_tree_height| (s.\ \ref{gn_tree_height}).

\subsection{Weitere Analyse und Visualisierung gespeicherter Stammb"aume}


\subsubsection{\texttt{gn\_ostream\_ddistribution\_rle}}

\begin{verbatim}
extern double gn_ddc(const GN_TREE *gtree, long generation, ostream &s);
\end{verbatim}

Diese Funktion gibt die Verteilung der phylogenetischen Distanzen
(vgl.\ \ref{gn_ddc}) auf dem durch das dritte Argument angegebenen
Ausgabe-\textsl{stream} aus. Die ersten beiden Argumente sind wie bei
\verb|gn_ddc| (s.\ \ref{gn_ddc}). Die Distanzverteilung wird dabei
durch \textsl{run length encoding} komprimiert, indem bei einer Folge
von $n$ aufeinanderfolgenden gleichen H"aufigkeitswerten $x$ nicht
$n$-mal die Zahl $x$, sondern die Zahlen $-n$ und $x$ ausgegeben werden
(bei $n > 2$).


\subsection{Weitere Analyse und Visualisierung gespeicherter Stammb"aume}

Zur weiteren Verarbeitung von Stammb"aumen, die mittels des in \ref{function_overview}
angesprochenen Operators zum Abspeichern in Files geschrieben wurden, stehen folgende
Programme zur Verf"ugung:

\begin{enumerate}

\item \verb|gnhx|: Extrahiert die genealogischen Linien aller rezenten Individuen

\item \verb|gnps|: Erzeugt PostScript-Graphiken der Stammb"aume, wobei die einzelnen
    Knoten in Abh"angigkeit von aus dem Genom entnommenen Werten grau get"ont
    werden k"onnen.

\item \verb|gnps_tragic|: Erzeugt ebenfalls PostScript Graphiken von Stammb"aumen,
    bei dem die Graut"onung in Abh"angigkeit von der potentiellen H"ohe
    erfolgt.

\item \verb|strattree|: Ein Script, das aus einem Stammbaumfile mithilfe der
    o.g.\ Programme verschiedene PostScript-Graphiken sowie eine Reihe von
    mit \verb|gnuplot| visualisierbaren Daten erzeugt.

\end{enumerate}

Bei allen genannten Programmen kann jedes File, das einen Stammbaum enth"alt,
als Eingabe verwendet werden. Im Speziellen k"onnen die mit \verb|SGM3D| erzeugten
\textsl{resolution files} direkt verarbeitet werden. Dies ist deshalb m"oglich,
weil beim Einlesen alle Daten bis zum Auftreten einer Stammbaumanfangskennung
ignoriert ("`weggelesen"') werden. Daten, die dem abgespeicherten Stammbaum
folgen, werden gar nicht erst eingelesen. Daher ist es derzeit nicht m"oglich,
Files mit mehreren Stammb"aumen zu verarbeiten, dies kann jedoch durch relativ
einfache Erweiterungen des Codes bei Bedarf erreicht werden. Alle genannten
Programme erwarten einige auf der Kommandozeile "ubergebene Argumente in fester
Reihenfolge, wobei in einigen F"allen beim Fehlen der letzten Argumente \textsl{defaults}
verwendet werden. Das erste Argument ist stets der Name des Files, das den
Stammbaum enth"alt. Die Verwendung der Programme wird im Folgenden beschrieben.


\subsubsection{gnhx}
\label{gnhx}

\begin{verbatim}
gnhx <gntree file> <output file basename>
\end{verbatim}

Dieses Programm erzeugt aus dem Stammbaum f"ur jedes rezente Individuum eine
Genealogieliste. Diese enth"alt f"ur das betreffende Individuum sowie f"ur alle
seine Vorfahren jeweils eine Zeile, in der der Zeitschritt der Entstehung (Geburt),
die ID des Individuums sowie sein Genom enthalten sind. F"ur jedes Individuum wird
ein solches File angelegt. Die Namen dieser Files haben die Form
\verb|basename_|\textsl{nnnn}\verb|.gls|, wobei \textsl{nnnn} f"ur eine
vierstellige Dezimalzahl steht. Ein solches \textsl{"`genealogy listing"'}
aus \verb|SGM3D| hat die folgende Form:

\begin{verbatim}
# genealogy list of individual "g2957-0" from file FirstTest.r3000
2957 (ID: g2957-0): A0: -9.89313, A1: 4.7706, A2: -0.5624, R2H: 0.95600
2848 (ID: g2848-0): A0: -9.89313, A1: 4.9109, A2: -0.5624, R2H: 0.95600
2782 (ID: g2782-0): A0: -9.89313, A1: 4.9109, A2: -0.5624, R2H: 0.93834
2724 (ID: g2724-0): A0: -9.53186, A1: 4.9109, A2: -0.5624, R2H: 0.93834
2624 (ID: g2624-0): A0: -9.53186, A1: 5.0907, A2: -0.5624, R2H: 0.93834
2595 (ID: g2595-0): A0: -9.80306, A1: 5.0907, A2: -0.5624, R2H: 0.93834
2560 (ID: g2560-0): A0: -9.80306, A1: 5.2289, A2: -0.5624, R2H: 0.93834
2517 (ID: g2517-0): A0: -10.1994, A1: 5.2289, A2: -0.5624, R2H: 0.93834
2423 (ID: g2423-0): A0: -10.1994, A1: 5.2825, A2: -0.5624, R2H: 0.93834

[...]

\end{verbatim}

Diese Files k"onnen zur Vermittlung eines Eindrucks der Evolutionsdynamik dienen;
insbesondere eignen sie sich jedoch zur Weiterverarbeitung zu Datenfiles f"ur
\verb|gnuplot|.


\subsubsection{gnps}
\label{gnps}


\subsubsection{gnps\_tragic}
\label{gnps_tragic}


\end{document}

#ifndef H_GNLIB
#define H_GNLIB


#ifdef __cplusplus
extern "C" {
#endif

#include <iostream.h>
#include <fstream.h>
#include <stdio.h>  /* necessary for definition of FILE */
#include "gndefs.h"  // jan
#include "gntypes.h" // jan

/*
 * Functions to init and discard trees.
 */

extern int gn_free_tree(GN_TREE *gtree);
extern void gn_init_tree(GN_TREE *gtree);

/*
 * Compute genealogical distance
 */

extern long gn_distance(const GN_TREE *gtree, GN_NODE *p1, GN_NODE *p2, long g);

/*
 * Create new nodes and update upon death
 */

extern int gn_new_treenode(GN_TREE *gtree, const GN_NODE_ID *p_ancestor_id, long generation, const char *name, const char *genome, GN_NODE_ID *new_nid);
extern int gn_node_death(GN_TREE *gtree, const GN_NODE_ID *gn_nid, long death_generation);

/*
 * Generate unique string identifying a node
 */

extern char *gn_node_idstring(const GN_NODE_ID *id, char *buf);

/*
 * Copy a genealogy tree in memory
 */

extern int gn_copy_tree(GN_TREE *target, const GN_TREE *source);

/*
 * Print a PHYLIP compatible representation of current tree
 */

extern int gn_print_jftrees(const GN_TREE *gtree, long generation, FILE *f);
extern int gn_fprint_history(const GN_TREE *gtree, FILE *f, const GN_NODE_ID *id);
extern int gn_fprint_postscript(const GN_TREE *gtree, FILE *f, long generation, double width, double height, double fontheight, const char *label);

/*
 * Save and restore a tree to a file
 */

extern int gn_read_tree(GN_TREE *gtree, FILE *f);
extern int gn_save_tree(const GN_TREE *gtree, FILE *f);

/*
 * Save and restore node IDs to a file
 */

extern int gn_read_id(GN_NODE_ID *gn_nid, FILE *f);
extern int gn_save_id(const GN_NODE_ID *gn_nid, FILE *f);

/*
 * determine some statistics of tree
 */


extern long gn_tree_height(const GN_TREE *gtree, long generation);
extern long gn_tree_length(const GN_TREE *gtree, long generation);
extern double gn_ddc(const GN_TREE *gtree, long generation);

/*
 * dump node contents to stdout, for debugging purposes
 */

extern int gn_print_node(const GN_NODE *gn);
extern int gn_print_subtree(const GN_NODE *gn);
extern int gn_print_tree(const GN_TREE *gtree);

#ifdef __cplusplus
}

extern ostream &operator << (ostream &s, const GN_NODE_ID &nid);
extern istream &operator >> (istream &s, GN_NODE_ID &nid);
extern ostream &operator << (ostream &s, const GN_TREE &gn_tree);

extern istream &operator >> (istream &s, GN_TREE &gn_tree);

extern int gn_ostream_jftrees(const GN_TREE *gtree, long generation, ostream &s);
extern int gn_ostream_ddistribution_rle(const GN_TREE *gtree, long generation, ostream &s);

#endif


#endif

